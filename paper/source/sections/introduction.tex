\section{Introduction}

\textcolor{red}{Short motivation why machine learning is interesting in econometrics}
With the rise of cheap computing research and application of computer-aided statistical
algorithm have skyrocketed. Machine learning, a sub-area of computational statistics,
that lays specific focus on using complex algorithms has found itself a home at the
computer science department. Other research areas have tried to utilize methods
developed in the machine learning literature. Researchers in economics and econometrics
have tried to find subproblems that could use the methodology. One of the main problems
of using machine learning methods is that it is not straightforward to use them in
settings where parameter inference is needed. Recent developments in econometrics have
introduced estimators that require an estimation of nuisance functions where one is only
interested in the prediction quality. Given high-level assumptions on these estimators
one can utilize many different machine learning estimators that have a fast-enough
convergence to the true function. In this paper I will investigate the convergence
properties of one particular method, the feed-forward neural network.
\textcolor{red}{Cite Chernozhukov etc.}

The rest of the paper is structured as follows: In the remaining part of the
introduction I will provide a small introduction to neural networks. This part can be
skipped if the reader is familiar with neural networks. In section 2 I discuss fitting
procedures for neural networks. In particular, I will describe $\ell_1$ and $\ell_2$
regularization for the

\subsection{A Primer on Neural Networks}

In the remaining parts of this section I provide a small primer on feed-forward neural
networks. I discuss the connection between their graphical and mathematical
representation, as well as common concepts such as the activation function. Fitting
techniques are presented in the next section. For a detailed description of neural
networks see \cite{Goodfellow.2016, Murphy.2012, Hastie.2008}.

Another popular regression method now mainly developed in the machine learning
literature are \emph{feed-forward} neural networks. The approximation quality of neural
networks has been studied extensively in the literature. Several so called universal
approximation theorems have been proven which show that under certain conditions on what
is known as the \emph{architecture} of the network, neural networks can approximate any
continuous function arbitrary well (\cite{Hornik.1991}). Statements like the universal
approximation theorems and the empirical success of networks make them a suitable
surrogate model candidate. Neural networks as surrogate models have been used across a
wide variety of fields; see for example \cite{Holena.2010, Gang.2019, Tripathy.2018} to
name a few.

In the following I will shed light on the inner workings of feed-forward networks by
showing that networks can be formulated analytically using simple linear algebra. I will
also comment on the backpropagation algorithm which provides a way to estimate the
parameters of a network using data. For a detailed description of neural networks and
the backpropagation algorithm see \cite{Goodfellow.2016, Murphy.2012, Hastie.2008}.

\paragraph{General Notion}

To understand how a statistical / machine learning method works one has to understand at
least two separate parts. First, given a fixed set of parameters how does the method
produce predictions, and second, how does the method estimate this set of parameters
given data. Note also that there must not be only one estimation step. Linear models can
be estimated using ordinary least squares but also through regularized approaches. To
better understand how neural networks work let us ignore the estimation component for
now.

\begin{figure}[!ht]
\centering
\begin{tikzpicture}[shorten >=1pt,->,draw=black!60, node distance=\layersep, scale=1.2]

    \tikzstyle{every pin edge}=[<-,shorten <=1pt]
    \tikzstyle{neuron}=[circle, draw=black!80, minimum size=17pt,inner sep=0pt]
    \tikzstyle{input neuron}=[neuron];
    \tikzstyle{output neuron}=[neuron];
    \tikzstyle{hidden neuron}=[neuron];
    \tikzstyle{annot} = [text width=4em, text centered]

    % Draw the input layer nodes
    \foreach \name / \y in {1,...,4}
    % This is the same as writing \foreach \name / \y in {1/1,2/2,3/3,4/4}
        \node[input neuron, pin=left:Input \y] (I-\name) at (0,-\y) {};

    % Draw the first hidden layer nodes
    \foreach \name / \y in {1,...,5}
        \path[yshift=0.5cm]
            node[hidden neuron] (H-\name) at (\layersep,-\y cm) {};

    % Draw the second hidden layer nodes
    \foreach \name / \y in {6/1,7/2,8/3,9/4,10/5}
        \path[yshift=0.5cm]
            node[hidden neuron] (H-\name) at (2 * \layersep,-\y cm) {};

    % Draw the output layer node
    \node[output neuron,pin={[pin edge={->}]right:Output}, right of=H-8] (O) {};

    % Connect every node in the input layer with every node in the
    % hidden layer.
    \foreach \source in {1,...,4}
        \foreach \dest in {1,...,5}
            \path (I-\source) edge (H-\dest);

    % Connect every node in the first hidden layer with every node in the
    % second hidden layer.
    \foreach \source in {1,...,5}
        \foreach \dest in {6,...,10}
            \path (H-\source) edge (H-\dest);

    % Connect every node in the hidden layer with the output layer
    \foreach \source in {6,...,10}
        \path (H-\source) edge (O);

    % Annotate the layers
    \node[annot,above of=H-1, node distance=1cm] (hl1) {Hidden layer 1};
    \node[annot,above of=H-6, node distance=1cm] (hl2) {Hidden layer 2};
    \node[annot,left = 1.75cm of hl1] {Input layer};
    \node[annot,right = 0.5cm of hl2] {Output layer};
\end{tikzpicture}
\caption{Depiction of a neural network. The network has 4 inputs, two hidden layers
with 5 nodes each and a single output node.}
\label{fig:neural_net}
\end{figure}

Consider the depiction of a two-layer network in figure \ref{fig:neural_net}. This
neural network represents a function $g : \mathbb{R}^4 \to \mathbb{R}$. The input layer
simply represents the input vector (feature vector) of the relation we wish to design /
approximate. The edges connect the nodes between the layers. When considering the
mathematical formulation we will see that the edges represent the weights of how much a
node influences the results in a target node. At last, the output node stores the actual
output. When modeling a function with multiple outcomes the output layer would have the
respective number of nodes.

To get started with the mathematical formulation focus on the connection between the
input layer and a single hidden node, as visualized in figure
\ref{fig:neural_net_single_connection}.

\begin{figure}
\centering
\begin{tikzpicture}[shorten >=1pt,->,draw=black!25, node distance=\layersep, scale=1.5]

    \tikzstyle{every pin edge}=[<-,shorten <=1pt, draw=black!90]
    \tikzstyle{neuron}=[circle, draw=black!80, minimum size=17pt,inner sep=0pt]
    \tikzstyle{input neuron}=[neuron, draw=black!90];
    \tikzstyle{output neuron}=[neuron];
    \tikzstyle{hidden neuron}=[neuron, draw=black!90];
    \tikzstyle{hidden neuron background}=[neuron, draw=black!25];
    \tikzstyle{annot} = [text width=4em, text centered]

    % Draw the input layer nodes
    \foreach \name / \y in {1,...,4}
    % This is the same as writing \foreach \name / \y in {1/1,2/2,3/3,4/4}
        \node[input neuron, pin=left:Input \y] (I-\name) at (0,-\y) {$x_\name$};

    % Draw the first hidden layer nodes
    \path[yshift=0.5cm]
        node[hidden neuron] (H-2) at (\layersep,-2 cm) {$h$};
    \foreach \name / \y in {1,3,4,5}
        \path[yshift=0.5cm]
            node[hidden neuron background] (H-\name) at (\layersep,-\y cm) {};

    % Connect every node in the input layer with every node in the
    % hidden layer.
    \foreach \source in {1,...,4}
        \foreach \dest in {1,3,4,5}
            \path (I-\source) edge (H-\dest);
    \foreach \source in {1,...,4}
        \path (I-\source) edge[draw=black!90] (H-2);

    % Annotate the layers
    \node[annot,above of=H-1, node distance=1cm] (hl1) {Hidden layer 1};
    \node[annot,left = 1.75cm of hl1] {Input layer};
\end{tikzpicture}

\caption{Depiction of the connection between the input layer and a single hidden node.}
\label{fig:neural_net_single_connection}
\end{figure}

Let me call the second node in the hidden layer in figure
\ref{fig:neural_net_single_connection} node $h$. The state of node $h$, that is, the
numerical value it stores after the network receives an input $x = (x_1, x_2, x_3,
x_4)^\top$, is simply given by
\begin{align*}
    h(x) = \sigma(w^\top x) \,,
\end{align*}
where $w \in \mathbb{R}^4$ denote the weights which are represented by the edges in the
diagram and $\sigma$ denotes the so called \emph{activation function}. The idea of using
an activation function comes from the biological nature of neural networks, in that it
was observed that human neurons need a certain magnitude of input before firing
(\cite{Goodfellow.2016}). Traditionally the most common activation function is the
sigmoid function $\sigma: \mathbb{R} \to [0, 1], z \mapsto {1}/{(1 + e^z)}$ which is
depicted in figure \ref{fig:sigmoid_function}.

\begin{figure}[!ht]
\centering
\begin{tikzpicture}[scale=1]
    \begin{axis}[
        axis on top = true,
        axis x line = bottom,
        axis y line = left,
        xlabel = $z$,
        ylabel = $\sigma$
    ]
        \addplot[
            domain = -5:5,
            samples = 100
        ]
            {1/(1+exp(-x))};

    \end{axis}
\end{tikzpicture}
\caption{Sigmoid function.}\label{fig:sigmoid_function}
\end{figure}

To compute the state of the hidden node in the second layer we need more detailed
notation. Consider figure \ref{fig:neural_net_double_connection} and let me denote the
second node of the second hidden layer again by node $h$.
\begin{figure}[h!]
\centering
\begin{tikzpicture}[shorten >=1pt,->,draw=black!25, node distance=\layersep, scale=1.5]

    \tikzstyle{every pin edge}=[<-,shorten <=1pt, draw=black!90]
    \tikzstyle{neuron}=[circle, draw=black!80, minimum size=17pt,inner sep=0pt]
    \tikzstyle{input neuron}=[neuron];
    \tikzstyle{hidden neuron}=[neuron];
    \tikzstyle{hidden neuron background}=[neuron, draw=black!25];
    \tikzstyle{annot} = [text width=4em, text centered]

    % Draw the input layer nodes
    \foreach \name / \y in {1,...,4}
    % This is the same as writing \foreach \name / \y in {1/1,2/2,3/3,4/4}
        \node[input neuron, pin=left:Input \y] (I-\name) at (0,-\y) {$x_\name$};

    % Draw the first hidden layer nodes
    \foreach \name / \y in {1,...,5}
        \path[yshift=0.5cm]
            node[hidden neuron] (H-\name) at (\layersep,-\y cm) {$h_\name$};

    % Draw the second hidden layer nodes
    \path[yshift=0.5cm]
        node[hidden neuron] (H-7) at (2 * \layersep,-2 cm) {$h$};
    \foreach \name / \y in {6/1,8/3,9/4,10/5}
        \path[yshift=0.5cm]
            node[hidden neuron background] (H-\name) at (2 * \layersep,-\y cm) {};

    % Connect every node in the input layer with every node in the
    % hidden layer.
    \foreach \source in {1,...,4}
        \foreach \dest in {1,...,5}
            \path (I-\source) edge[draw=black!90] (H-\dest);

    % Connect every node in the first hidden layer with every node in the
    % second hidden layer.
    \foreach \source in {1,...,5}
        \path (H-\source) edge[draw=black!90] (H-7);
    \foreach \source in {1,...,5}
        \foreach \dest in {6,8,9,10}
            \path (H-\source) edge (H-\dest);

    % Annotate the layers
    \node[annot,above of=H-1, node distance=1cm] (hl1) {Hidden layer 1};
    \node[annot,above of=H-6, node distance=1cm] (hl2) {Hidden layer 2};
    \node[annot,left = 1.75cm of hl1] {Input layer};
\end{tikzpicture}

\caption{Depiction of the connection between the input layer and a single hidden node
in the second hidden layer through the first hidden layer.}
\label{fig:neural_net_double_connection}
\end{figure}

For each node $j=1,\dots,5$ in the first hidden layer we can compute the state using the
input vector $x$, the respective weights $w_j$ and the activation function $\sigma$.
This gives $h_j = \sigma(w_j^\top x)$ for each $j$. The values of the first hidden layer
can be combined in a vector
\begin{align*}
    \mathbf{h}(x) :=
        \begin{bmatrix}
            \sigma(w_1^\top x)\\
            \vdots\\
            \sigma(w_5^\top x)
        \end{bmatrix} \,,
\end{align*}
which can be further simplified by defining the weight matrix for the interactions
between the input-layer and first hidden layer as $W := (w_1 \dots, w_5)^\top$ and
allowing $\sigma$ to be extended to element-wise operations. Then
\begin{align*}
    \mathbf{h}(x) := \sigma(W x) \,.
\end{align*}
Because the internal structure of the network does not change the state of node $h$ is
then simply given by
\begin{align*}
    h(x) = \sigma(w_h^\top \mathbf{h}(x)) \,,
\end{align*}
where $w_h$ denote the weights from the nodes in the first hidden layer to node $h$.

Combining all of the above we can write out the analytical formulation of the output of
the complete two-layer network depicted in figure \ref{fig:neural_net}. For this define
$W_1$ as the weighting matrix from the input layer to the first hidden layer, $W_2$ as
the weighting matrix from the first hidden layer to the second hidden layer, and let
$w^*$ be the weighting vector from the last hidden layer to the final output node.
Using again that $\sigma$ can be applied element-wise we get
\begin{align*}
    g(x) &= w^* \mathbf{h}_2(x)\\
         &= w^* \sigma(W_2 \, \mathbf{h}_1(x))\\
         &= w^* \sigma \big( W_2 \, \sigma(W_1 \, x) \big) \,.\tag{Neural Network}
\end{align*}

\begin{remark}\hfill
    \begin{itemize}
        \item  To keep the discussion simple I ignored that in most cases each target
            node, say $h$, is computed with an intercept $w_0$, i.e. $h(x) = \sigma(w_0 +
            w^\top x)$. In the literature this is called the \emph{bias}.
        \item Note that the activation function is not applied to the output layer
            as it maps its input to $[0, 1]$ but we want it to map to $\mathbb{R}$.
        \item The number of hidden layers and the number of nodes per hidden layer are
            collectively called the architecture of the network. There is no one best
            architecture and scanning over all possible architectures is clearly
            intractable. Most commonly one chooses a few architectures and compares
            their performance using the methods proposed in section
            \ref{subsec:model_comparison}.
    \end{itemize}
\end{remark}

\paragraph{Extensions}

The easiest extension to the network I presented above is to use a different activation
function; see \cite{Hara.1994} for a comparison. A popular choice is the rectified
linear unit (ReLU) activation function, depicted in figure \ref{fig:relu_function}.
\cite{Hinton.2017} provide empirical evidence that the ReLU function can improve upon
the classical sigmoid function by accelerating the convergence of the gradient descent
step.

\begin{figure}[!ht]
\centering
\begin{tikzpicture}[scale=1]
    \begin{axis}[
        axis on top = true,
        axis x line = bottom,
        axis y line = left,
        xlabel = $z$,
        ylabel = $\sigma$,
        ymin=-0.5
    ]
    \addplot[
        domain=-3:5,
        samples=100
    ] {(x>=0)*x};
    \end{axis}
\end{tikzpicture}
\caption{Rectified linear unit (ReLU) activation function.}\label{fig:relu_function}
\end{figure}
