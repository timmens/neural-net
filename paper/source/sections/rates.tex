\section{Rates}
\label{seq:rates}

In this section I consider two different kind of results. First, I present
representation results that answer the question which classes of functions can be
approximated by a neural network. Second, I consider convergence results that answer
the question if a neural network can learn a function using data.

%%%%%%%%%%%%%%%%%%%%%%%%%%%%%%%%%%%%%%%%%%%%%%%%%%%%%%%%%%%%%%%%%%%%%%%%%%%%%%%%%%%%%%%%
% Approximation Theorems
%%%%%%%%%%%%%%%%%%%%%%%%%%%%%%%%%%%%%%%%%%%%%%%%%%%%%%%%%%%%%%%%%%%%%%%%%%%%%%%%%%%%%%%%
\subsection{Approximation Theorems}

When we use neural networks to estimate functional relationships we hope that the
network can approximate the true function as closely as possible. A minimal assumption
that can give us hope would be if we could be assured that there exists a neural
network that can approximate the target function with some degree of accuracy. The
well known Weierstra{\ss} theorem provides this result for any continuous real-valued
function $f$ defined on a closed domain $[a, b]$: For every $\epsilon > 0$ there exists
a polynomial $q_\epsilon$ such that $\sup_{x} |f(x) - q_\epsilon(x)| < \epsilon$; for
a proof consult any advanced real analysis textbook. Neural networks are more flexible
than polynomials in their architecture; one can alter the number of hidden neurons per
layer, the number of layers or the class of activation functions. Here I present two
different results. In the first we consider a class of networks with one hidden layer
that can grow arbitrarily large. In the second I consider a class of networks with fixed
width per layer that can grow arbitrarily deep. In both cases the result is that under
regularity conditions networks can approximate any continuous functions arbitrarily
well.

\begin{theorem}
   Let $\sigma : \mathbb{R} \to \mathbb{R}$ be any non-polynomial continuous function.
   Let $\mathbb{N}_p^\sigma$ represent the class of neural networks with actiation
   function $\sigma$, input dimension $p$ and one hidden layer with an arbitrary number
   of neurons. Then $\mathbb{N}_p^\sigma$ is dense in the space of continuous functions
   $[0, 1]^p \to \mathbb{R}$, with respect to the uniform norm.
\end{theorem}
\begin{proof}
    See \cite{Cybenko.1989,Hornik.1991,Pinkus.1999}.
\end{proof}

To get more intuition on why this result holds consider a one-layer network with $h$
hidden neurons
\[
    f(x) = a_0 + \sum_{l = 1}^h a_l \cdot \sigma(w_l^\top x + b_l).
\]
Now by cleverly setting the activation function and parameters we can reconstruct basis
functions. Letting $h$ tend to infinity, the asymptotic approximation quality should
therefore be the same as of other series approximators, e.g., the Fourier series.

\begin{theorem}
    Let $\sigma : \mathbb{R} \to \mathbb{R}$ be any non-affine continuous function such
    that there exists some $x \in \mathbb{R}$ for which $\sigma$ is continuously
    differentiable at $x$ with $\sigma'(x) \neq 0$. Let $\mathbb{N}_{p, h}^\sigma$
    denote the class of networks with activation functions $\sigma$, input dimension
    $p$ and an arbitrary number of hidden layers that each have $h$ number of neurons.
    Then $\mathbb{N}_{p,h}^\sigma$ is dense in the space of continuous functions
    $[0, 1]^p \to \mathbb{R}$ if $h \geq p + 3$, with respect to the uniform norm.
\end{theorem}
\begin{proof}
    See \cite{Kidger.2020}.
\end{proof}

Similar to the Weierstra{\ss} theorem and polynomials, these theorems motivate the usage
of neural networks for approximation tasks. However, existence results do not help when
choosing the network architecture, and more importantly, they do not show that a neural
network can actually learn a structure from data.

%%%%%%%%%%%%%%%%%%%%%%%%%%%%%%%%%%%%%%%%%%%%%%%%%%%%%%%%%%%%%%%%%%%%%%%%%%%%%%%%%%%%%%%%
% Convergence Rates
%%%%%%%%%%%%%%%%%%%%%%%%%%%%%%%%%%%%%%%%%%%%%%%%%%%%%%%%%%%%%%%%%%%%%%%%%%%%%%%%%%%%%%%%
\subsection{Convergence Rates}

In a strict sense neural networks are parametric estimators with (potentially) many
parameters. Assuming that the width or depth of a network can grow with the sample size
these estimators can be analyzed in a nonparametric setting. In nonparametric analysis
its common to assume that the target function is $k$-smooth. In this setting the optimal
minimax rate for the $L_2$ prediction error is $\mathcal{O}(n^{-2k/(2k+d)})$; see
\cite{Stone.1982}. The applications where neural networks had the biggest impact require
very large input dimensions $d$, e.g., image classifcation, where a feature vector can
consist of $512 \times 512$ pixels. In these scenarios even huge sample sizes are not
suffizient to compensate the slow rates. Thus, a natural questions is whether these
rates persist for neural networks and if they can be approved upon when imposing more
structure on the target function.

While these theoretical considerations would be by far the most interesting part,
discussing them in detail would go beyond the scope of this paper. Nevertheless, let me
present some recent papers tackling these questions.

\cite{Bauer.2019} show that if the target function satisfies a $(p, C)$-smooth
generalized hierarchical interaction model of given order $d^\ast$, for $d^\ast \leq d$,
a deep neural network estimator defined via the solution to the least-squares problem
\ref{eq:objective} can circument the curse-of-dimensionality and generate rates of the
form $\mathcal{O}(n^{-{2p}/{(2p + d^\ast)}})$.

In practice neural networks are not estimated by directly solving the minimization
problem \ref{eq:objective}, but through (derivations of) the gradient descent
algorithm; see Section \ref{seq:fitting}. \cite{Bauer.2019} do not account for this
different estimation scheme. \cite{Braun.2019} consider single-layer neural network
estimators where the weights are learned using a gradient descent procedure. Assuming
that the target functions follows a projection pursuit model with $(p, C)$-smooth base
functions, they show that a neural network fitted via gradient descent achieves an
$L_2$ prediction error rate
\[
\mathcal{O}\left(\frac{\log^3 n}{n} \right)^{\frac{2p}{2p + 1}}.
\]
Notably, this rate does not depend on the input dimension $d$. However, this is not
a surprising result, since the structure of the projection pursuit model is very
similar as the structure of a one-layer neural network.
