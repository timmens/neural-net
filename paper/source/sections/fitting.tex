\section{Fitting}

In this section I discuss general fitting techniques with a focus on estimators that
induce a non-convex loss function. I explain approaches used when training neural
networks, and how they are extended to regularize the fitting process. I assume that an
IID dataset $\{(x_i, y_i) | i = 1, \dots, n\}$ is available with features $x_i \in
\mathbb{R}^p$ and conditional expectation $f(x) = \mathbb{E}[y_i | x_i = x]$. If not
otherwise stated I assume that that all necessary moments exist. The problem I consider
in this paper is to solve is to find a \emph{good} approximation of the conditional
expectation $f$ using the data. In section \textcolor{red}{Introduction} I argued that
this problem arises naturally in econometrics when estimating nuisance functions that
are usually conditional expectations.

\subsection{General Fitting Techniques}

\textcolor{red}{
Let us consider a generic class of approximation functions $\{g \in C_{a.e.}^1 |
g(\cdot) = g(\cdot; \theta, \eta), \theta \in \Theta\}$}\footnote{By $C_{a.e.}^1$ I
denote the space of all functions $\mathbb{R}^p \to \mathbb{R}$ that are at least once
differentiable almost-everywhere with respect to the Lebesgue measure.}, with $\theta$
denoting a given parametrization and $\eta$ a hyper-parameter configuration. This
distinction is necessary since hyper-parameters are not estimated during the fitting
process. For neural networks a typical hyper-parameter is the network architecture and
the class of activation functions. Similarly, for boosting the class of weak-learners
and the learning rate are viewed as fixed during the estimation. Grid-based methods like
cross-validation and similar traditional approaches can be used to estimate such
hyper-parameters on the data; however, these methods quickly become infeasible if there
are more than two variables to calibrate. This is due to the fact that the number of
grid-points grows exponentially with the dimensions, making the problem computationally
intractable. Modern approaches utilize Bayesian optimization techniques to deal with
high-dimensional hyper-parameter spaces; see for example \textcolor{red}{BABLAB}. In the
following I will assume a fixed hyper-parameter configuration, and for the sake of
clarity, I will thus suppress any dependence on $\eta$.

\paragraph{Loss function}

What expression do we need to minimize to get an estimator of $\theta$? From statistical
decision theory we know that the conditional expectation $f$ solves the problem
\[
    \min_h \mathbb{E}\left[\ell\left(y_i, h(x_i)\right) \right],
\]
when $\ell(y, \hat{y}) = (y - \hat{y})^2$ is the squared error loss function. Given our
class of approximation functions we can parametrize this problem via $\theta$
\[
    \min_{\theta \in \Theta} \mathbb{E}\left[\ell\left(y_i, g(x_i; \theta)\right) \right].
\]
Because we don't know the joint distribution of $(x_i, y_i)$ this object is intractable;
however, we can use a sample analogue or non-linear least-squares approach to define
the estimator as the solution to a minimization problem containing only observed
quantities
\begin{align}
    \min_{\theta \in \Theta} \sum_{i = 1}^n \ell\left(y_i, g(x_i; \theta)\right).
    \tag{MP}
    \label{eq:objective}
\end{align}
This defines the empirical loss function $L(\theta) = \sum_{i = 1}^n \ell\left(y_i,
g(x_i; \theta)\right)$ that we wish to minimize. Define $\hat{\theta} := \text{argmin}
\, L(\theta)$ as the parameter that solves \ref{eq:objective}, and $\hat{f}(x) := g(x;
\hat{\theta})$ as the estimator of $f$. Notice that we can make two main mistakes during
the estimation. First, because we work on a finite sample $\hat{\theta}$ will not
minimize the expected loss, and second, $f$ may not be representable using the class of
approximation functions. In section \textcolor{red}{Approximation theorems} I discuss
results on the representation properties of neural networks, indicating that for large
enough models the latter problem can be neglected.

\paragraph{Minimization}

How do we minimize $L(\theta)$? An optimal strategy depends on the class of
approximation functions. In the linear case $g(x; \theta) = x^\top \theta$ we know that
$L$ is convex and that its minimizer corresponds to the ordinary least-squares
estimator. In the general case $g$ may induce $L$ to be non-convex. Thus, there may not
be an analytical solution and there may be multiple local minima. In these settings one
commonly uses iterative procedures of the form

\begin{algorithmic}[1]
\State $\theta_0 \gets$ initial value
\While{not converged}\\
$\theta_{k+1} \gets O(\theta_k, L)$
\EndWhile
\end{algorithmic}
That is, we choose an initial starting value of the parameter and then iteratively apply
an updating procedure $O$ to the parameter. The procedure, or optimizer, uses the
current parameter state $\theta_k$ and the loss function $L$. If we assume more
structure on $L$ we can use procedures with faster convergence rates. A common
assumption is differentiability, in which case the procedure uses gradient information
during the updating step. The most known case being the gradient descent algorithm, that
performs the following updating rule
\begin{align}
    O(\theta_k, L) = \theta_k - \eta_k \nabla L(\theta_k),
    \tag{GD}
\end{align}
where $\eta_k > 0$ governs the learning speed. Many of the modern optimization
procedures derive from the gradient descent algorithm, such as stochastic gradient
descent or Adam; see \textcolor{red}{BALBLA}. These optimizers are well studied and
convergence proofs exist under many different sets of assumptions on $L$ and
$\{\eta_k\}$. For example, convergence to a local minimum is achieved if $L$ is twice
continuously differentiable with Lipschitz Hessian; see \textcolor{red}{BLUBU}.

Note that contrary to intuition using second-order information does not necessarily
improve the procedure. For large models the Hessian matrix of $L$ can be very expensive
to compute, as closed-form expressions are rarely available. In this case the gain in
accuracy has to offset the loss in computational time. This is why optimizers using
second-order information are rarely used when training models with many parameters, in
comparison to small-scale models, like logistic regression which is commonly solved
using the second-order Newton-Raphson algorithm; see \textcolor{red}{BLABS}.


\subsection{Fitting of Neural Networks}

The complexity of training a neural network can be attributed to two main problems. One,
the (computationally efficient) calculation of the gradient $\nabla L(\theta)$, and two,
the avoidance of overfitting.

\paragraph{Backpropagation}

As we have seen above, computation of the gradient $\nabla L(\theta)$ is essential to
train a neural network. Since neural networks are constructed using a composition of
nonlinear activation functions applied to matrix-vector products, the structure of the
derivative can be derived using the chain-rule. Unfortunately, naive implementations of
this procedure are so inefficient that they cannot be used to train large neural
networks. The term backpropagation was coined by \cite{Rumelhart.1986}, who proposed a
computationally efficient algorithm for the calculation of the gradient. For a detailed
explanation of the backpropagation technique and its connection to automatic
differentiation see section 6.5 of \cite{Goodfellow.2016}.

\paragraph{Regularization}

The number of parameters in a neural network can explode quickly. The network depicted
in figure \ref{fig:neural_net} contains $4 \times 5 + 5 \times 5 + 5 \times 1 = 50$
parameters (excluding the biases). In the case of a 100-dimensional feature vector and
two hidden layers with 100 hidden nodes each, the number of parameters would grow to
$100 \times 100 + 100 \times 100 + 100 = 20100$. Clearly overfitting is an acute hazard
in large networks. Many methods have been developed to mitigate this problem, such as
stochastic gradient descent (\cite{Ruppert.1985}), dropout regularization
(\cite{Srivastava.2014}), early stopping (\cite{Caruana.2000}) or weight decay
(\cite{Krogh.1991}).

In my simulations I use two different approaches to fit neural networks. In the first,
which I call \emph{unregularized}, I use the Adam optimizer that applies a similar
method as stochastic gradient descent. In the second, which I call \emph{regularized}, I
use the same optimizer and additionally place an $\ell_1$-penalty on the weights
connecting the input layer and the first hidden layer. This approach is similar to
weight-decay, which uses an $\ell_2$-penalty on all weights.
