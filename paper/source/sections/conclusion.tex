\section{Conclusion}
\label{seq:conclusion}

In this paper, I tried to give an overview of neural networks and their fitting
procedures in the context of econometrics. The simulation study provided evidence that
neural networks can outperform linear models in specific scenarios but also showed that
other machine learning algorithms can easily beat them. Considering under which
circumstances neural networks are successful in the industry, it seems that they require
a large number of training samples and architectural fine-tuning by machine learning
engineers. In this light, neural networks are not algorithms that one passes a data set
to and expects good results. Achievement of good results requires the correct setting of
hyper-parameters and optimizers. The theory on the convergence of neural networks has
just started considering some of these options; e.g., \cite{Braun.2019}. However, the
gap between what is done in practice and what is analyzed theoretically remains large.
