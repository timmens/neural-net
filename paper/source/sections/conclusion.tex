\section{Conclusion}
\label{seq:conclusion}

In this paper I tried to give an overview of neural networks and their fitting
procedures in the context of econometrics. The simulation study provided evidence that
neural networks can outperform linear models in certain scenarios, but also showed
that they can easily be beaten by other machine learning algorithms. Considering when
neural networks are successful in the industry, it seems that they require large number
of training samples and architectural fine-tuning by the machine learning engineers.
Under this light, neural networks are not an algorithm that one passes a data set to and
expects good results. Achievement of good results requires the correct setting of
hyper-parameters and optimizers. The theory on convergence of neural networks has just
started considering some of these options; e.g., \cite{Braun.2019}. But the gap between
what is done in practice and what is analyzed theoretically remains large.
